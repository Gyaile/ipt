% !TeX root = cours-ipt.tex
\renewcommand{\FrenchLabelItem}{\textbullet}

\chapter{Listes}

\section{Types composés}
Jusqu’à présent nous avons utilisé des données simples : 

entiers, réels, booléens.

On a souvent besoin de manipuler des données multiples, on a alors besoin de définir des assemblages de types simples :

\begin{itemize}
\item une suite de résultats expérimentaux
\item un répertoire de noms
\item une fiche d’un répertoire avec noms, téléphone, mail, etc
\item un bulletin de notes
\item une file d’attente
\item et bien d’autres
\end{itemize}

Les assemblages peuvent avoir différentes propriétés :
\begin{itemize}
\item longueur fixée (ensemble des notes trimestrielles) ou variable (résultats expérimentaux)
\item type composant homogène (liste de références) ou non (fiche de répertoire)
\item accès au composants par leur place (suite d’entiers),  par un nom (intitulés d’un répertoire) ou accès d’un élément à la fois (file d’attente)
\item modification possible d’un élément (moyenne des devoirs des étudiants) ou modification impossible (caractéristiques d’un atome).
\end{itemize}

Chaque langage définit des objets composés en faisant des choix, en général plusieurs types sont proposés. Python définit en particulier les listes : ce sont des suites de composants de types quelconques, indexés par leur position, de longueur variable et dont les éléments sont modifiables. 

On peut imaginer les listes comme un train de marchandises : chaque wagon a son numéro, on peut en remplacer le contenu, on peu ajouter un ou des wagons au bout. On peut aussi, mais c’est plus long, insérer un wagon au milieu ; on doit alors renuméroter tous les wagons qui suivent.

\section{Listes comme tableaux}

On va commencer par voir comment utiliser une liste comme un tableau de variables.
Ce sera la principale utilisation.

\subsection{Définition d'une liste}

On peut définir une liste en l’écrivant explicitement :
\begin{lstlisting}
>>> maListe = [2,7,8]
>>> maListe
[2,7,8]
\end{lstlisting}


On peut aussi définir une liste à partir d’un générateur, comme dans la boucle for
\begin{lstlisting}
>>> maListe = list(range(20)) 
>>> maListe
[0,1,2,3,4,5,6,7,8,9,10,11,12,13,14,15,16,17,18,19]
\end{lstlisting}

\textbf{Rappel} : en Python 2 le générateur est \lstinline?xrange?, 

par contre \lstinline?range? définit directement une liste.

\subsection{Accès aux éléments}

Les éléments de la liste sont obtenus par leur indice sous la forme
\begin{lstlisting}
>>> maListe[1]
7
\end{lstlisting}

On remarquera que les indices commencent à 0.

Python permet aussi d’accéder aux éléments depuis la fin : le dernier élément a l’indice -1, le précédent l’indice -2 et on continue jusqu’au premier qui a l’indice -n où n est la longueur de la liste.


\[
\begin{tabular}{cccccccc}
éléments & $\vert$ & 4&8&7&5&6&7\\
  \hline
indice & $\vert$ & 0&1&2&3&4&5\\
  \hline
indice négatif& $\vert$ & -6&-5&-4&-3&-2&-1\\
\end{tabular}
\]


On peut modifier un élément avec les mêmes moyens :
\begin{lstlisting}
>>> maListe[1]=5
>>> maListe[-1]=3
>>> maListe
[2, 5, 3]
\end{lstlisting}

Les listes sont donc des structures mutables, on verra que Python définit des structures qui ne permettent pas la modification, non mutables.

\subsection{Longueur d'une liste}

Le nombre d’éléments d’une liste, sa longueur, est donné par 
\begin{lstlisting}
>>> len(maListe)
3
\end{lstlisting}

\lstinline?len? signifie length, longueur en anglais.

Comme le premier indice d'une liste est 0 le dernier sera $n-1$ où $n$ est la longueur.


\section{Parcours d’une liste}

On aura souvent besoin de parcourir une liste c’est-à-dire inspecter ses éléments un à un.
Pour cela une boucle for est bien adaptée car les indices forment une suite d’entiers.

On va traiter un exemple : recherche du maximum d'une liste.

\subsection{Parcours standard}

On cherche la valeur maximale dans une liste d’entiers ou de réels.
L’idée est de trouver, pas à pas, le maximum des premiers termes ; 
si on connaît le maximum des termes d’indices 0 à i il suffit de le comparer avec le terme d’indice i+1 pour choisir le maximum des termes d’indices 0 à i+1.

\begin{lstlisting}
def maxListe1(l):
	""" Recherche du maximum de la liste l
	Celle-ci doit etre non vide"""
    maximum = l[0]
    for i in range(len(l)):
        if maximum < l[i]:
            maximum = l[i]
    return(maximum)
\end{lstlisting}

\break

Sous cette forme on compare, à la première itération, l[0] avec lui-même. Bien que ce ne soit pas vraiment un problème on peut l’éviter.
\begin{lstlisting}
def maxListe2(l):
	maximum = l[0]
	for i in range(1,len(l)):
		if maximum < l[i]:
			maximum = l[i]
	return(maximum)
\end{lstlisting}


\subsection{Parcours pythonesque}
On accède aux différentes valeurs de la liste par leur indice. Python permet d’éviter cela et de parcourir directement la liste, c’est une évolution de la boucle spécifique à ce langage.

\begin{lstlisting}
def maxListe3(l):
	maximum = l[0]
	for x in l:
		if maximum < x:
			maximum = x
	return(maximum)
\end{lstlisting}


Il y a des cas où cette méthode peut ne pas sembler possible.
Par exemple si on souhaite non pas la valeur maximale mais l’indice en lequel cet indice est atteint.
Pour cela on doit modifier le programme en gardant en mémoire le maximum provisoire mais aussi l’indice correspondant.
\begin{lstlisting}
def indiceMaxListe1(l):
	""" Recherche du premier indice en
	lequel la liste atteint son maximum"""	
	maximum = l[0]
	iMaximum = 0
	for i in range(len(l)):
		if maximum < l[i]:
			maximum = l[i]
			iMaximum = i
	return(iMaximum)
\end{lstlisting}

\break

Ici encore Python fournit un autre moyen.
La fonction {\tt enumerate} accède aux couples (i,x) formés des différents indices successifs et aux valeurs correspondantes de la liste :
\begin{lstlisting}
def indiceMaxListe2(l):
	iMaximum,maximum = 0,l[0]
	for i,x in enumerate(l):
		if maximum < x:
			iMaximum,maximum = i,x
	return(iMaximum)
\end{lstlisting}


\section{Constructions de nouvelles listes}


\subsection{Concaténation}

Les opérations des chaînes de caractères s’appliquent aux listes. On construit de nouvelles listes à partir de listes existantes.

\begin{lstlisting}
>>> l = [1,5,4,8]
>>> ll = [2,4,7]
>>> l+ll
[1,5,4,8,2,4,7]
>>> l*3
[1,5,4,8,1,5,4,8,1,5,4,8]
>>> l*0
[]
\end{lstlisting}


\subsection{Extraction}

L’extraction de tranche se fait en indiquant le premier indice (0 par défaut), le premier indice non pris (on prend jusqu'à la fin par défaut) et le pas.


\begin{lstlisting}
>>> l = [3,1,4,1,5,9,2,6,5,2]
>>> l[2:7]
[4,1,5,9,2]
>>> l[:-3] # on enleve les 3 derniers termes
[3,1,4,1,5,9,2]
>>> l[2::3]# 1 terme sur 3 a partir de l[2]
[4, 9, 5]
\end{lstlisting}

Python permet d’utiliser cette construction dès la définition d’une liste. 
On remplace \lstinline?list(range(n))[a:n:b]? par \lstinline?list(range(a,n,b))?

\subsection{Application d'une fonction}

On peut aussi appliquer une fonction à tous les éléments d’une liste. On construit une nouvelle liste.
\begin{lstlisting}
>>> l = [1,5,4,8]
>>> ll = [x*x for x in l]
>>> ll
[1,25,16,64]
\end{lstlisting}

On peut même appliquer une fonction aux éléments d’une liste après filtrage :
\begin{lstlisting}
>>> l = [1.1, 0, 0.4, 7.4, 0.18]
>>> ll = [1/x for x in l if abs(x) > 0.001]
>>> ll
[0.9090909090909091, 2.5, 0.13513513513513511, 5.555555555555555]
\end{lstlisting}

\section{Modification sur place}

Les listes, comme tout ce qui est employé dans Python, sont des objets. 

Un objet définit des fonctions associées, \textbf{les méthodes} qu'on lui applique par l’appel d’une procédure de la forme
\textit{nomDeListe.methode(parametres)}.

Ceci ne produit pas une nouvelle liste mais modifie celle à laquelle est appliquée la méthode.



La méthode principale est append(x) qui ajoute x au bout de la liste :
\begin{lstlisting}
>>> l = [1,5,4,8]
>>> l.append(3)
>>> l
[1,5,4,8,3]
\end{lstlisting}

C’est un bon moyen pour fabriquer une liste en ajoutant des termes successivement.
\begin{lstlisting}
def chercher(x,l):
	"""chercher les occurrences de x dans l
	le resultat est la liste des indices 
	i tels que l[i]=x"""
	resultat=[] 
	for i,y in enumerate l:
		if x == y:
			resultat.append(i) 
	return(resultat) 
\end{lstlisting}


D’autres méthodes, qui transforment la liste sans en fabriquer de nouvelle, sont :

\begin{itemize}
\item l’insertion, \lstinline?l.insert(k,x)? insère l’élément x à la k-ième position en décalant d’un cran les éléments suivants.
\item la concaténation, \lstinline?l.extend(ll)? transforme l en l + ll
\item la suppression, \lstinline?l.remove(x)? supprime la première occurrence de l’élément x dans la liste en décalant d’un cran les éléments suivants
\end{itemize}

\section{Structure de liste}

\subsection{Organisation en mémoire}

Les variables de type listes ne sont pas organisées comme le sont les types simples. La raison principale est que la liste est modifiable 

Une instruction de la forme \lstinline?a=3? construit un emplacement mémoire, que l'on se représentera sous la forme d'une case, occupé par l'entier 3 à une adresse que l'ordinateur calcule à partir du nom {\tt a}, que l'on se représentera sous la forme d'une étiquette.

$$\tikzpicture
 \tikzstyle{every node}=[circle,draw,minimum size =10mm]
 \tikzstyle{level 1}=[sibling distance =6cm]
 \node {\tt a}
      child {node[rectangle]{2}};
\endtikzpicture
$$

L'affectation \lstinline?b=a? consiste à créer une nouvelle case en y copiant le contenu de la première en lui attribuant l'étiquette {\tt b}.

$$\tikzpicture
 \tikzstyle{every node}=[circle,draw,minimum size =10mm]
 \tikzstyle{level 1}=[sibling distance =6cm]
 \node {\tt a}
      child {node[rectangle]{2}};
 \node at (3,0) {\tt b}
      child {node[rectangle]{2}};
\endtikzpicture
$$

Lors de la création d'une liste la case ne contient pas la liste mais un indicateur vers la liste.

$$\tikzpicture
 \tikzstyle{every node}=[circle,draw,minimum size =10mm]
 \tikzstyle{level 1}=[sibling distance =6cm]
 \node {\tt l}
      child {node[rectangle](sl){}};
 \draw (2,-2) grid +(5,1);
 \draw[thick,-stealth] (sl) -- (2,-1.5);
\endtikzpicture
$$

\subsection{Affectation}


L’affectation d’une liste \textbf{ne crée pas} une nouvelle liste :
\begin{lstlisting}
>>> l1 = [1,5,4,8]
>>> l2 = l
>>> l1[1]=3
>>> l2
[1,3,4,8]
\end{lstlisting}

En effet les deux cases contiennent l'indicateur de la liste.

$$\tikzpicture
 \tikzstyle{every node}=[circle,draw,minimum size =10mm]
 \tikzstyle{level 1}=[sibling distance =6cm]
 \node at (-3,0) {\tt l1}
      child {node[rectangle](al){}};
 \draw (0,-4) grid +(5,1);
 \draw[thick,-stealth] (al) -- (0.5,-3);
 \node at (3,0) {\tt l2} 
      child {node[rectangle](all){}};
 \draw[thick,-stealth] (all) -- (0.5,-3);
\endtikzpicture
$$



Il faut penser que \lstinline?l1? et \lstinline?l2? sont deux noms pour le même objet. Pour reprendre l’analogie ferroviaire si \lstinline?l1? est le train numéro 44844, \lstinline?l2? peut être “le train Hazebrouck-Lille de 19h27”. 
Le nom d’une liste ne réfère pas à la liste mais à un indicateur de l’endroit où est la liste.

On peut penser cela aussi comme deux noms associés à la même case.

$$\tikzpicture
 \tikzstyle{every node}=[circle,draw,minimum size =10mm]
 \node (l1) {\tt l1};
 \node (l2) at (1.5,0) {\tt l2};
 \draw(-0.5,-2) rectangle (2,-1);
 \draw (l1) -- (0,-1);
 \draw (l2) -- (1.5,-1);
 \draw (4,-2) grid +(5,1);
 \draw [thick,-stealth] (2,-1.5) -- (4,-1.5);
\endtikzpicture
$$


\subsection{Copie}On veut cependant parfois créer une nouvelle liste en copiant une liste existante de telle manière que les deux listes soient ensuite indépendantes. Plusieurs moyen sont possibles

\begin{itemize}

\item La fonction \lstinline?list? crée une liste  à partir d'objets énumérés tels que les listes :

\begin{lstlisting}
>>> l1 = [1,5,4,8]
>>> l2 = list(l1)
\end{lstlisting}

\item On peut extraire la première liste en entier :

\begin{lstlisting}
>>> l1 = [1,5,4,8]
>>> l2 = l1[:]
\end{lstlisting}


\item On peut utiliser un cas particulier d'affectation de fonction avec l'identité :

\begin{lstlisting}
>>> l1 = [1,5,4,8]
>>> l2 = [x for x in l]
\end{lstlisting}

\item On peut utiliser \lstinline?deepcopy? que l'on charge depuis le module \lstinline?copy?

\begin{lstlisting}
>>> from copy import deepcopy
\end{lstlisting}

On verra plus loin que cette fonction va plus loin que le permettent les autres méthodes.

\begin{lstlisting}
>>> l1 = [1,5,4,8]
>>> l2 = deepcopy(l1)
\end{lstlisting}



\end{itemize}

Dans chaque cas on obtient deux listes indépendantes.

\begin{lstlisting}
>>> l2
[1,5,4,8]
>>> l1[1]=3
>>> l1
[1,3,4,8]
>>> l2
[1,5,4,8]
\end{lstlisting}


\section{Listes de listes}

\subsection{Définition}

Les éléments d'une liste ne sont pas nécessairement de type simple, en particulier on peut y placer d'autres listes. On obtient ainsi un tableau à double entrée.

On peut placer des listes de taille différentes (et de types différents) mais nous n'allons considérer que des listes à $n$ composantes chacune étant une liste de $p$ composantes, des matrices.

\begin{lstlisting}
>>> m = [[1,5,4,8],[2,5,1,2],[3,0,2,7]]
\end{lstlisting}

$$\tikzpicture
 \tikzstyle{every node}=[draw,minimum size =10mm,fill=white]
 \node[circle] (m) {\tt m}
  child {node[rectangle](l){}};
 \node[rectangle](l0) at (3,-1.5) {};
 \node[circle](ll0) at (3,-1.5) {};
 \node[rectangle](l1) at (4,-1.5) {};
 \node[circle](ll1) at (4,-1.5) {};
 \node[rectangle](l2) at (5,-1.5) {};
 \node[circle](ll2) at (5,-1.5) {};
 \draw [thick,-stealth] (l) -- (l0);
 \node[rectangle](m20) at (9,-5) {3};
 \node[rectangle](m21) at (10,-5) {0};
 \node[rectangle](m22) at (11,-5) {2};
 \node[rectangle](m23) at (12,-5) {7};
 \draw [thick,-stealth] (l2) -- (m20);
 \node[rectangle](m10) at (4,-5) {2};
 \node[rectangle](m11) at (5,-5) {5};
 \node[rectangle](m12) at (6,-5) {1};
 \node[rectangle](m13) at (7,-5) {2};
 \draw [thick,-stealth] (l1) -- (m10);
 \node[rectangle](m00) at (-1,-5) {1};
 \node[rectangle](m01) at (0,-5) {5};
 \node[rectangle](m02) at (1,-5) {4};
 \node[rectangle](m03) at (2,-5) {8};
 \draw [thick,-stealth] (l0) -- (m00);
\endtikzpicture$$



\vskip 1cm

On peut penser \lstinline?m? sous la forme
$\begin{pmatrix}
1&5&4&8\\ 
2&5&1&2\\ 
3&0&2&7\\
\end{pmatrix}$.

Les éléments sont alors accessibles par \lstinline?m[i][j]?

\begin{lstlisting}
>>> m[0][2]
4
\end{lstlisting}

\subsection{Initialisation}

Pour initialiser une matrice nulle on peut être tenté de construire ainsi 

\begin{lstlisting}
>>> r = [0,0,0]
>>> m =[r,r,r]
>>> m
[[0, 0, 0], [0, 0, 0], [0, 0, 0]]
\end{lstlisting}


Mais voici ce qui se passe quand on modifie un terme

\begin{lstlisting}
>>> m[0][1] = 1
>>> m
[[0, 1, 0], [0, 1, 0], [0, 1, 0]]
\end{lstlisting}

Ce qui s'est passé est que chacune des lignes de la matrice est une case associée à la liste \lstinline?r? on a donc modifié cette liste qui est reproduite 3 fois

\begin{lstlisting}
>>> r
[0, 1, 0]
\end{lstlisting}

$$\tikzpicture
 \tikzstyle{every node}=[draw,minimum size =10mm]
 \node[circle] (m) {\tt m}
  child {node[rectangle](l){}};
 \node[rectangle](l0) at (3,-1.5) {{\tt r}};
 \node[circle](ll0) at (3,-1.5) {};
 \node[rectangle](l1) at (4,-1.5) {{\tt r}};
 \node[circle](ll1) at (4,-1.5) {};
 \node[rectangle](l2) at (5,-1.5) {{\tt r}};
 \node[circle](ll2) at (5,-1.5) {};
 \draw [thick,-stealth] (l) -- (l0);
 \node[rectangle](r0) at (4,-3.5) {0};
 \node[rectangle](r1) at (5,-3.5) {1};
 \node[rectangle](r2) at (6,-3.5) {0};
 \draw [thick,-stealth] (l0) -- (r0);
 \draw [thick,-stealth] (l1) -- (r0);
 \draw [thick,-stealth] (l2) -- (r0);

\endtikzpicture
$$


Il faut donc recopier, par un des moyens vus ci-dessus, la liste ligne.

\begin{lstlisting}
>>> r = [0,0,0]
>>> matrice =[list(r),list(r),list(r)]
>>> matrice
[[0, 0, 0], [0, 0, 0], [0, 0, 0]]
>>> matrice[0][1] = 1
>>> matrice
[[0, 1, 0], [0, 0, 0], [0, 0, 0]]
\end{lstlisting}

\subsection{Copie}
La copie d'une matrice pose des problèmes plus ardus que dans le cas d'une liste.
En effet si on utilise les premiers moyens du paragraphe précédent on ne copie pas les éléments mais les listes qui forment les lignes.

\begin{lstlisting}
>>> mm = [x for x in m]
\end{lstlisting}

$$\tikzpicture
 \tikzstyle{every node}=[draw,minimum size =10mm,fill=white]
 \node[circle] (m) {\tt m}
  child {node[rectangle](l){}};
 \node[rectangle](l0) at (3,-1.5) {};
 \node[circle](ll0) at (3,-1.5) {};
 \node[rectangle](l1) at (4,-1.5) {};
 \node[circle](ll1) at (4,-1.5) {};
 \node[rectangle](l2) at (5,-1.5) {};
 \node[circle](ll2) at (5,-1.5) {};
 \draw [thick,-stealth] (l) -- (l0);
 \node[rectangle](m20) at (9,-5) {3};
 \node[rectangle](m21) at (10,-5) {0};
 \node[rectangle](m22) at (11,-5) {2};
 \node[rectangle](m23) at (12,-5) {7};
 \draw [thick,-stealth] (l2) -- (m20);
 \node[rectangle](m10) at (4,-5) {2};
 \node[rectangle](m11) at (5,-5) {5};
 \node[rectangle](m12) at (6,-5) {1};
 \node[rectangle](m13) at (7,-5) {2};
 \draw [thick,-stealth] (l1) -- (m10);
 \node[rectangle](m00) at (-1,-5) {1};
 \node[rectangle](m01) at (0,-5) {5};
 \node[rectangle](m02) at (1,-5) {4};
 \node[rectangle](m03) at (2,-5) {8};
 \draw [thick,-stealth] (l0) -- (m00);
 \node[circle] (mm) at (0,-7) {\tt mm}
  child {node[rectangle](r){}};
 \node[rectangle](r0) at (3,-8.5) {};
 \node[circle](rr0) at (3,-8.5) {};
 \node[rectangle](r1) at (4,-8.5) {};
 \node[circle](rr1) at (4,-8.5) {};
 \node[rectangle](r2) at (5,-8.5) {};
 \node[circle](rr2) at (5,-8.5) {};
 \draw [thick,-stealth] (r) -- (r0);
 \draw [thick,-stealth] (r2) -- (m20);
 \draw [thick,-stealth] (r1) -- (m10);
 \draw [thick,-stealth] (r0) -- (m00);
 \endtikzpicture
$$

Ici le seul outil est \lstinline?deepcopy? qui copiera chaque élément de la liste en appliquant non pas une affectation mais encore \lstinline?deepcopy? : c'est une fonction \textbf{récursive}, létude de la récursivité se fera en seconde année.

\vskip 2cm

Plus tard nous utiliserons le module \lstinline?numpy? qui implémente de manière très efficace les matrices.

